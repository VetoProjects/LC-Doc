\tableofcontents
\newpage
\section{Einleitung}
\epigraph{If debugging is the process of removing software bugs, then programming must be the process of putting them in.}{Edsger W. Dijkstra}
\paragraph{}
    Im Folgenden wird eine Live-Coding-Applikation vorgestellt, die im Rahmen einer an der Hochschule für Technik und Wirtschaft
	Berlin angebotenen Lehrveranstaltung des Bachelorstudiengangs Angewandte Informatik, Vertiefung Multimedia, entwickelt wurde.
	Das Projektziel war frei wählbar, die Anforderung war jedoch, dass es sich um eine Multimediaanwendung handelt, die Gebrauch von
	der Programmbibliothek QT\footnote{http://qt-project.org/} macht und in C++, der ursprünglichen und nativen Sprache von QT, 
	geschrieben ist\footnote{Obwohl es inzwischen verschiedenste Wrapper und Erweiterungen gibt. Siehe \ref{sec:QT} auf 
	Seite \pageref{sec:QT}}.
\paragraph{}
	Die Autoren entschieden sich für eine Live-Coding-Applikation\footnote{Für eine Definition des Begriffs wird auf \ref{sec:LiveCoding}
	auf Seite \pageref{sec:LiveCoding} verwiesen}, da diese eine breite Einführung in die Möglichkeiten von QT bietet und es momentan
	nach ihrem Wissenstand keine anderen adäquaten Tools für dieses Subset des Programmierens gibt. Ausserdem besteht bei den Autoren
	ein grosses Interesse an Metaprogrammierung und dem Mechanismus der Quelltextübersetzung sowie für die Sache des Live Coding.
	Somit schufen sie sich auch ein Werkzeug für ihre eigenen Zwecke, das jedoch auch zum freien Download auf Github
	angeboten wird. An dieser Stelle besteht mit einiger Wahrscheinlichkeit bereits ein Bedarf nach Definition, dem in \ref{sec:LiveCoding}
	und \ref{sec:QT} nachgegangen wird.
\paragraph{}
    \textit{Eine aktuelle Version des Werkzeuges kann zu jedem Zeitpunkt von der Seite }\url{https://github.com/hellerve/VetoLC}\textit{ abgerufen werden.
    Ebenso kann die aktuellste Version dieser Dokumentation von der Webseite}\linebreak
    \url{https://github.com/hellerve/LiveCoding-Documentation}\textit{ oder auf Anfrage von den Autoren bezogen werden.}
\newpage
\subsection{Gliederung der Arbeit}
	Im Folgenden wird die Gliederung der Arbeit beschrieben; sie spiegelt den Entwicklungsprozess wider:
\paragraph{Abschnitt \ref{sec:Def}}
	beschäftigt sich mit den grundlegenden Begrifflichkeiten der vorliegenden Arbeit und gibt eine kurze, nicht umfassende Einführung in QT als
	Bibliothek.\footnote{Eine solche Einführung könnte leicht eine eigene Arbeit füllen.}
\paragraph{Abschnitt \ref{sec:Ziel}}
	behandelt die Zielsetzung des Projektes. Sie behandelt die Planung sowie die implementationsunabhängigen Vorüberlegungen der Autoren.
	Es wird ein Überblick über die Infrastruktur und ihr zugrundeliegende Design-Entscheidungen gegeben.
\paragraph{Abschnitt \ref{sec:Imp}}
	beschreibt die Implementation und geht auf technische Details ein. Der Entwicklungsprozess und die einzelnen Teile des Programms werden 
	erläutert. Dies ist natürlich nicht in aller Tiefe möglich, jedoch ergänzt sich die vorliegende Arbeit hierin mit der Dokumentation des 
    Programms.
\paragraph{Abschnitt \ref{sec:WhatsNext}}
	stellt Fragen, die nach der ersten Version der Entwicklungsumgebung offen bleiben, denn die Autoren planen, daran weiterzuarbeiten.
\paragraph{Abschnitt \ref{sec:Schluss}}
	versucht Schlussworte zu finden. Dies ist derjenige Teil der Arbeit, der den Autoren am schwersten von der Hand ging.
\newpage
\section{Definitionen} \label{sec:Def}
	Im Folgenden werden einige für das Verständnis des weiteren Textes elementare Definitionen bereitgestellt.
\subsection{Definition des Begriffes Live Coding} \label{sec:LiveCoding}
	Live Coding ist eine neue und relativ interessante Disziplin der Informatik, die dem künstlerischen Potential der Programmierung von
	Multimedia-Anwendungen noch die Kraft der Performance-Kunst, die Kraft des Moments, eröffnet. Durch entsprechende Werkzeuge soll das 
	Schreiben von Anwendungen in Echtzeit zur Kunstform erhoben werden. Dejaying\footnote{hier: das Erstellen von audiomanipulativen Programmen 
	in Echtzeit, seit kurzem bekannt als \textit{Algorave}.} ist genauso vom Begriff eingeschlossen wie Grafikprogrammierung, wobei diese vor allem von 
	Programmen wie Processing\footnote{http://www.processing.org}, einer Entwicklungsumgebung für Java, dominiert wird. In der Klangprogrammierung
	gibt es keine solch dominante Kraft, die verschiedensten Sprachen und Umgebungen werden benutzt, von Tidal\footnote{http://toplap.org/tidal/}, einer
	in Haskell implementierten funktionalen Nischensprache, bis hin zu Super Collider\footnote{http://supercollider.sourceforge.net/}, einer mächtigen 
    objektorientierten Sprache. Wir hoffen dass wir uns durch die Auswahl Pythons, einer objektorientierten Sprache, die jedoch auch viele funktionale
    Elemente enthält(\texttt{map()}, \texttt{filter()}, \texttt{lambda}, Generatoren, Closures), ein möglichst gutes Fundament erwählt haben, um beide Ansätze 
    zumindest zu emulieren.
\paragraph{}
	Durch die Erschaffung einer hinreichend komfortablen, mächtigen Entwicklungsumgebung mit einem breiten Spektrum an Möglichkeiten erhoffen sich 
	die Autoren ein erhöhtes Interesse an der Sache des Live Coding sowie erhöhte Produktivität der kreativ involvierten Entwickler und Künstler; eine
	solche Plattform sollte aus ihrem Verständnis heraus von Grund auf neu geschrieben werden, da die Umformung bestehender Umgebungen meist
	ungewollte Kompromisse mit sich zieht, die das Projekt entweder stagnieren lassen oder in ein anderes, ungewolltes Produkt umwandeln würden.
\newpage
\subsection{Einführung in die Programmbibliothek QT} \label{sec:QT}
\epigraph{More computing sins are committed in the name of efficiency (without necessarily achieving it) than for any other single reason – including blind stupidity.}
{William A. Wulf}
	QT(ausgesprochen \textit{kju:t}\footnote{Die Kriptik der IPA-Zeichen begeistert zumindest einen der Autoren zutiefst.}, wie das englische Wort 
	\textit{cute}) ist eine im Jahr 1991 von zwei schwedischen Entwicklern konzipierte Programmbibliothek, die plattformübergreifende Programmierung 
	von grafischen Oberflächen, Netzwerk- und Datenbankprogrammen und Spielen ermöglicht, sequentiell und nebenläufig. Sie basiert auf dem Verständnis 
    von Ereignissen, Signalen und Slots sowie den Erweiterungen von C++, die durch einen zusätzlichen Präprozessor, den Meta Object Compiler(moc),
	ermöglicht werden. 
\paragraph{}
	Klassen, die diese Funktionen nutzen wollen, müssen das Schlüsselwort \linebreak\texttt{Q\_OBJECT} einbinden und zusätzlich zu den 
	\texttt{private} und \texttt{public}-Teilen der Klassendefinition noch \texttt{(private/public) slot} und \texttt{signal} definieren.
\subsubsection{Ereignisse in QT}
	Ereignisse werden in einer von einer Qt-Applikation(QApplication) begonnenen Ereignisschleife(Event Loop) verarbeitet. Passiert nichts, läuft die 
	Schleife endlos weiter\footnote{Vereinfacht ausgedrückt könnte man von Busy Waiting sprechen, jedoch ist die interne Struktur der Event Loop sehr viel
	komplexer, siehe \label{sec:evloop}}. Sie wird durch ein Schliessereignis(Close Event) unterbrochen. Auch das Erzeugen neuer asynchron 
	laufender Schleifen ist möglich(Dazu müssen bestimmte Objekte - \texttt{QThread}, \texttt{QProcess}, \texttt{QThreadpool} etc. - erstellt und die 
    laufenden Prozesse dorthin verschoben werden).
\subsubsection{Die Ereignisschleife} \label{sec:evloop}
	Wenn ein Programmierer die Ereignisschleife nach den Standardvorgaben startet, ruft er \texttt{QApplication::exec()} auf. Diese leitet ihn dann
	durch zwei \texttt{exec()}-Definitionen der Basisklassen zu \texttt{QtCoreApplication}, die  testet, ob das Aufrufen der Schleife valide ist und dann eine
	 Instanz der internen Klasse \texttt{QEventLoop} erstellt. Von dieser wiederum \texttt{exec()} aufgerufen\footnote{Zur Erinnerung: Dies ist die vierte
	Klasse, die diese Methode implementiert!}. Diese Methode ruft \linebreak \texttt{processEvents()} auf, die dann wartet, bis etwas passiert. Also doch ein Fall von
	Busy Waiting? Nein, die Methode bricht die Operation ab, wenn nach zu langer Zeit nichts passiert. Natürlich wird die Applikation dann nicht einfach abgebrochen,
	sondern eine weitere Schleifenkondition inder Methode darüber getestet, doch schlägt diese fehl(ein internes atomares Flag namens exit, das zeigt, dass,
	nun ja, ein Beenden des Programmes nötig wurde), bricht die Ereignisschleife ab. Der beinhaltende Thread  wird dann normalerweise beendet(außer man
	überschreibt gewisse Methoden in QThread, aber außer den Autoren dieser Arbeit braucht das heutzutage niemand mehr).
\subsubsection{Signale}
	Signale sind auf Ereignissen basierende Nachrichten, die von Objekten an die Ereignisschleife gereicht werden und von dort an die entsprechenden Slots
	versendet(durch Broadcasting an alle verbundenen Slots) werden.
\subsubsection{Slots}
	Slots sind Routinen, die an bestimmte Signale gebunden werden können und müssen, um auf sie zu reagieren. Der Absender muss zur 
	Definitionszeit bekannt sein, das heisst, es ist nich möglich, einfach Signale von allen oder unbekannten Objekten entgegenzunehmen.
\paragraph{}
	Slots können als \texttt{private} oderm \texttt{public} deklariert sein und damit entweder interne oder externe Sichtbarkeit besitzen.
\subsubsection{Der Meta Object Compiler(moc)}
	Der Meta Object Compiler übersetzt den nicht-C++-konformen QT-Code in C++-Code, indem er alle Makros und die Definition von Signalen und Slots
	durch C++-Code ersetzt. Dieser (heute unötig) komplexe Vorgang ist der Historie der Bibliothek geschuldet, da diese in einer Zeit entstand, in der
	Kompilatoren die generische Programmierung durch Vorlagen(Templates) nicht gleich gut unterstützen. Dies wiederum stand in krassem Gegensatz
	zum plattformunabhängigen Ansatz QTs.
\newpage
\section{Zielsetzung} \label{sec:Ziel}
\epigraph{Walking on water and developing software from a specification are easy if both are frozen.}{Edward V Berard}
	Eine Entwicklungsumgebung zu erschaffen, die den Ansprüchen einer Entwicklergemeinde gerecht wird, die auf hohe Verfügbarkeit und Robustheit bei 
    gleichzeitiger Garantie der Evaluation in Echtzeit aufbaut, ist eine Aufgabe, der sich die Autoren nicht gewachsen fühlen. Kompensiert wird dies
    durch die offene Lizensierung, die den Benutzer dazu aufruft, selbst an dem Programm zu arbeiten. Trotzdem sind, um den Grundstein einer solchen
    Entwicklung zu legen, einige Vorüberlegungen nötig. Die Infrastruktur des Werkzeugs lässt sich im Nachhinein nur unter grossen Schwierigkeiten 
    wieder ändern und meist schreibt ein engagierter Entwickler dann lieber ein neues, besseres Programm\footnote{So entstand zumindest die Idee der
    Autorn zu ihrem Editor.}.
\subsection{Allgemeine Infrastruktur}
    War der erste Prototyp des Editors noch relativ monolithisch, wurde mit seiner Erweiterung und der Verbesserung der Funktionalität klar,
    dass dies keine zukunftsträchtige oder elegante Lösung war. Dazu sei auch noch gesagt, dass die erste Skizze der Infrastruktur bereits modular
    war, jedoch waren diese guten und wohlgeformten Ideen in der ersten Euphorie vergessen. Die letztendlich verwendete Infrastruktur sieht jener
    Skizze doch wieder ähnlich; ein Beweis dafür, dass Ideen meist besser sind als ihre erste Umsetzung in die Tat dies vermuten lässt.
\paragraph{}
    Auf Abbildung \ref{figure:figureI} ist die grundlegende Infrastruktur des Programmes dargestellt. Alle dem Benutzer bekannten Instanzen sind weiss
    hinterlegt, alle ihm nicht notwendigerweise erfassbaren grau. Der Kontrollfluss ist relativ klar: Der Benutzer interagiert nur mit dem Editor,
    dieser mit den Einstellungen, um auf diese zu reagieren) und dem Backend, um Informationen über Eingaben des Benutzers weiterzugeben und auf Nachrichten
    der geöffneten Kontexte und des Systems zu reagieren. Das Backend wiederum ist ebenfalls lesend mit den Einstellungen verbunden. An dieser Stelle muss
    eine erste Anmerkung gemacht werden: die Instanz Einstellungen ist komplett ideell, d.h. es gibt \textit{keine} ihr entsprechende Implementation.
\paragraph{}
    Weiterhin ist das Backend mit dem den Code ausführendem Kontext verbunden. Diese sind nicht identisch, der Kontext ist eine asynchrone Instanz,
    die nach Bedarf gestartet und gestoppt wird(d.h. ein Thread). Dies erlaubt zum einen die weitere Benutzung des Editors, auch wenn gerade ein Programm
    ausgeführt wird und die Kapselung und Abstraktion. Die Steuerung des Kontextes erfolgt durch Signale durch das Backend.
\paragraph{}
    In der Realität ist die Infrastruktur sehr viel komplexer und verzweigter, doch die grundlegenden Komponenten sind diejenigen, die in Abbildung
    \ref{figure:figureI} zu sehen sind; meist sind die eigentlich zusammenhängenden Ideen noch in Unterprobleme aufgespalten und gekapselt.
\begin{figure}
  \centering
  \hspace{3cm}
  \begin{psmatrix}[mnode=r,colsep=0.8,rowsep=0.5]
    [name=IO-Device] \pw{IO-Device} & [name=User] \pw{User} \\[0pt]
    [name=Renderer] \pw{Renderer} & [name=Execution Context] \ps{Execution Context} & [name=Backend]  \ps{Backend} & [name=Editor] \pw{Editor} & [name=Settings] \ps{Settings}
    \ncline{->}{User}{Editor}
    \ncline{->}{Editor}{Backend}
    \ncline{->}{Backend}{Editor}
    \ncline{->}{Editor}{Settings}
    \ncline{->}{Backend}{Execution Context}
    \ncline{->}{Execution Context}{IO-Device}
    \ncline{->}{Execution Context}{Renderer}
    \ncangle[angleA=-90,angleB=-90]{->}{Settings}{Backend}
  \end{psmatrix}
  \caption{Eine Übersicht über die Infrastruktur des Editors.}
\end{figure}
\label{figure:figureI}
\subsection{Entscheidungen im Vorfeld und ihre Begründung}
    Über Modularität und Kapselung lässt sich streiten, in vielen Fällen gab es durchaus zweierlei Wege, die Infrastruktur anzulegen, doch war eine Entscheidung
    vonnöten. Trotzdem fühlen die Autoren sich verpflichtet, Auskunft über die Gedanken zu geben, die hinter dem Aufbau stecken.
\paragraph{}
    So stellt sich die Frage, ob die Einstellungen tatsächlich vom Backend zu trennen sind. In der Ansicht der Autoren sind sie das, da sie Anweisungen des Benutzers
    sind, die zwar Anweisungen des Backends an den Editor entsprechen, jedoch nicht wirklich aus dem Arbeitsgang des Programmes stammen, sondern quasi in dieses injeziert
    werden. Desweiteren ist eine so einfache Kapselung kostengünstig und für die Wartbarkeit von Debeutung. So wird dem aufmerksamen Leser des Quelltextes aufgefallen sein, 
    dass neben dem Backend auch ein SettingsBackend existiert, das mit dem Backend in Verbindung steht. Dies ist eine sehr einfache, der Lesbarkeit jedoch zuträgliche
    Trennung und nur einer von einigen Punkten, in dem der kleine Überblick der Realität nicht ganz getreu wird.
\paragraph{}
    Ebenso ist der ausführende Kontext(im Überblick englisch betitelt), eigentlich ein Teil des Backends, jedoch läuft dieser asynchron und unabhängig vom Backend und
    steht mit diesem nur über \texttt{Signals} in Verbindung. Dies ist infrastrukturell geschickter, da es eine vom Hauptprogramm unabhängige Fehlerbehandlung im Kontext
    ermöglicht und diesem zusätzlich mehr CPU-Zeit zur Verfügung stellen sollte.\footnote{Dies ist eine blosse Vermutung, ungetestet und höchstwahrscheinlich falsch.}
\paragraph{}
    Abschliessend möchten die Autoren noch zugeben, dass sämtliche Infrastruktur in einem Versuch-und-Fehler-Arbeitsprozess entstand; dies lässt sich leicht anhand der
    Commit-Beschriftungen(und deren Inhalt) im Versionsrevisionswerkzeug(git) ablesen. Von einer Studie derselben raten die Verantwortlichen jedoch heftigst ab, da sich
    ein Moloch der schlechtmöglichsten Praktiken abzeichnet(so wurde zum Beispiel über den ganzen Erarbeitungsprozess nur eine einzige Branch bearbeitet).
\newpage
\section{Implementation} \label{sec:Imp}
	Die Beschreibung der Implementation ist nicht ganz einfach und es lohnt sich, die mithilfe Doxygen erstellte Klassendokumentation zu Rate zu ziehen. Glücklicher\-weise
    sind sämtliche Komponenten jedoch mehr oder weniger aussagekräftig dokumentiert, sodass ein geneigter Leser sich schnell zurecht finden sollte.
\section{Offene Fragen} \label{sec:WhatsNext}
	Was bleibt noch zu tun
\section{Schlussworte} \label{sec:Schluss}
	\epigraph{Most good programmers do programming not because they expect to get paid or get adulation by the public, but because it is fun to program.”}{Linus Torvalds}
